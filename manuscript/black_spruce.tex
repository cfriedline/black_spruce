\documentclass[11pt]{article}
\pdfoutput=1
\usepackage{simplemargins}
\usepackage{url}
\usepackage[pdftex]{graphicx}
\usepackage{setspace}
\graphicspath{{figures/}}
\usepackage{siunitx}
\setlength{\parindent}{0pt} 
\setlength{\parskip}{1.6ex} 
\setallmargins{1in} 
\linespread{1.6}
\usepackage[round]{natbib}
\usepackage{color}
\usepackage{subfigure}
\usepackage{booktabs}
\usepackage{pdflscape}
\usepackage[colorlinks=false,breaklinks]{hyperref}
\listfiles

% make subfigure labels capitalized
\renewcommand{\thesubfigure}{(\Alph{subfigure})}

%setup supplement
\newcommand{\beginsupplement}{%
        \setcounter{table}{0}
        \renewcommand{\thetable}{S\arabic{table}}
        \setcounter{figure}{0}
        \renewcommand{\thefigure}{S\arabic{figure}}
        \renewcommand{\thesection}{S\arabic{section}}
        \renewcommand{\thesubsection}{S\arabic{subsection}} 
     }

\begin{document}

% title must be 150 characters or less
\begin{flushleft} 
\singlespacing
{\large \textbf{title}}

% insert author names, affiliations and corresponding author email.
author1$^{1}$, 
author2$^{2}$,


\bf{1} affil 1
\\
\bf{2} affil 2
\\

$\ast$ e-mail: corresponding\_author@domain.com
\end{flushleft}

\section*{Abstract}

\section*{Introduction}

\section*{Materials and methods}\label{ss:mats}
A total of 7232 files chromatogram files from two sets of replicate
samples, two each for needle and cambium were processed using an
IPython \citep[v 2.1]{per-gra:2007} notebook and various utilities.
The chromatograms were processed using Phred \citep[v.\
020425]{ewing1998b} into fasta sequences with a trimming cutoff of
0.01, and sequences which were less than 100 bases long were excluded.
The ESTs were combined into single files according to tissue source
and replicate name (e.g., P32C, P32N, P40C, P40N).  Each of these
combined files were processed using SeqClean \citep[v.\
2/2011][default parameters]{citeulike:1911083}, screening both for
vectors using the UniVec \citep{cochrane01012010} database and
contamination from \textit{E. coli} K-12 substr.\ DH10B.  The cleaned
and trimmed ESTs were assembled into unigenes using iAssembler
\citep[v.\ 1.3.2][default parameters]{zheng2011iassembler} for each library.

The assembled unigenes were aligned to a local copy of the nr database
(downloaded in February 2014) with BLASTX \citep{citeulike:238188}
keeping, at most, the top 10 hits with an e-value cutoff of
$e^{-5}$. The resulting alignment xml files were filtered using
BioPython \citep[v.\ 1.6.4]{citeulike:4202607} to include all hits
which were at least 50\% of the unigene length and identity.  These
filtered BLAST results were imported into Blast2GO Pro \citep[v.\
2.7.2, bg2\_may14]{citeulike:2733895} for functional annotation.
Annotations were assigned using default parameters for the annotation
step (e-value: $e^{-6}$, annotation cutoff: 55, go weight: 5, HSP-hit
coverage cutoff: 0, taxonomy filtering: none) and evidence code
weights. The full suite of InterPro \citep{citeulike:12942060}
mappings were performed for each unigene with a valid BLAST result and
these results were merged with the GO annotations.  These annotations
were augmented with ANNEX \citep{annex} and mapped to enzyme codes and
KEGG \citep{citeulike:9172127} pathways. GO term encrichment for both
needle and cambium was tested directionally using Fisher's exact test
as implemented in Blast2GO.  Terms were considered to be significantly
enriched if the $p$-value of the test statistic was $<0.05$ while
controlling for False Discovery Rate
\citep[FDR,][]{benjamini1995controlling}.

All relevant code for both the analysis and manuscript can be found at 
\url{http://www.github.com/cfriedline/black_spruce}

\section*{Results}

\section*{Discussion}

\section*{Acknowledgements}

\clearpage

\singlespacing
\bibliographystyle{spbasic}
\bibliography{refs}

\clearpage
%tables

\clearpage
%figures

\clearpage

\beginsupplement

\section*{Supplementary material}

%put supplementary figures and tables below here
\subsection*{Supplementary Text}\label{ss:supp}


\end{document}
