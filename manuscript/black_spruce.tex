\documentclass[11pt]{article}
\pdfoutput=1
\usepackage{simplemargins}
\usepackage[pdftex]{graphicx}
\usepackage{setspace}
\graphicspath{{figures/}}
\usepackage{siunitx}
\setlength{\parindent}{0pt} 
\setlength{\parskip}{1.6ex} 
\setallmargins{1in} 
\linespread{1.6}
\usepackage[round]{natbib}
\usepackage{color}
\usepackage{subfigure}
\usepackage{booktabs}
\usepackage{pdflscape}
\usepackage{tabularx}
\usepackage[colorlinks=false]{hyperref}
\usepackage[anythingbreaks]{breakurl}
\usepackage{longtable}
\usepackage[tableposition=top]{caption}
\listfiles

% make subfigure labels capitalized
\renewcommand{\thesubfigure}{(\Alph{subfigure})}

%setup supplement
\newcommand{\beginsupplement}{%
        \setcounter{table}{0}
        \renewcommand{\thetable}{S\arabic{table}}
        \setcounter{figure}{0}
        \renewcommand{\thefigure}{S\arabic{figure}}
        \renewcommand{\thesection}{S\arabic{section}}
        \renewcommand{\thesubsection}{S\arabic{subsection}} 
     }

\begin{document}
\newdimen\LTcapwidth \LTcapwidth=\textwidth


% title must be 150 characters or less
% \begin{flushleft} 
% \singlespacing
% {\large \textbf{title}}

% insert author names, affiliations and corresponding author email.
% author1$^{1}$, 
% author2$^{2}$,


% \bf{1} affil 1
% \\
% \bf{2} affil 2
% \\

% $\ast$ e-mail: corresponding\_author@domain.com
% \end{flushleft}

% \section*{Abstract}

% \section*{Introduction}

\section*{Materials and methods}\label{ss:mats}
A total of 7232 chromatogram files from two sets of replicate samples,
two each for needle and cambium were processed using an IPython
\citep[v 2.1]{per-gra:2007} notebook and various utilities.  The
chromatograms were processed using Phred \citep[v.\
020425]{ewing1998b} into Fasta sequences with a trimming cutoff
probability of 0.01 (e.g., Phred score = 20), and sequences which were
less than 100 bases long were excluded.  The ESTs from all replicates
were combined into single file (where all sequence identifiers
indicated the source and replicate of the EST).  This combined file
was process first using SeqClean \citep[v.\ 2/2011][default
parameters]{citeulike:1911083}, screening both for vectors using the
UniVec \citep{cochrane01012010} database and contamination from
\textit{E. coli} K-12 substr.\ DH10B.  The cleaned and trimmed ESTs
were in each case were subsequently assembled into unigenes using
iAssembler \citep[v.\ 1.3.2][default parameters]{zheng2011iassembler}.

The assembled unigenes were aligned to a local copy of the nr database
(downloaded in February 2014) with BLASTX \citep{citeulike:238188}
keeping, at most, the top 10 hits with an e-value cutoff of
$e^{-5}$. The resulting alignment XML file were filtered using
BioPython \citep[v.\ 1.6.4]{citeulike:4202607} to include only the top
HSPs (i.e., alignments) which were at least 30\% of both unigene
length and identity as well as only those hits which aligned to a
sequence assigned to the NCBI taxonomic division, ``Plants''. These
filtered BLAST results were imported into Blast2GO Pro \citep[v.\
2.7.2, bg2\_sep14]{citeulike:2733895} for functional annotation using
the default parameters (e.g., keeping at most 10 hits with an HSP
length cutoff of 33 amino acids).  Annotations were assigned using
default parameters (e.g., e-value: $e^{-6}$, annotation cutoff: 55, go
weight: 5, HSP-hit coverage cutoff: 0, taxonomy filtering: none) and
evidence code weights. The full suite of InterPro
\citep{citeulike:12942060} mappings were performed for each unigene
with a valid BLAST result and these results were merged with the GO
annotations.  These annotations were augmented with ANNEX
\citep{annex} and mapped to enzyme codes and KEGG
\citep{citeulike:9172127} pathways. The resulting annotations were
then exported to a text file for further analysis. 

Differential gene (unigene) expression (DGE) was computed in R
\citep[v. 3.1.1]{R} using RSEM (RNA-Seq by Expectation-Maximization)
\citep[v. 1.2.15]{Li:BmcBioinformatics:2011}, a software package for estimating
gene expression levels from sequence data.  The input to the RSEM pipeline was a
multi-FASTA file of assembled unigenes.  Bowtie indices
\citep[v. 2.2.3]{Langmead:NatMethods:2012} against these files were built as
part of the data preparation process.  For each individual (e.g., P32 and P40)
and for each tissue (e.g., cambium and needle, combined by parent), the gene
expression levels were calculated. In all cases, the models were evaluated using
the built-in RSEM diagnostics.  The RSEM procedure resulted in a unigene
$\times$ individual matrix, where the vales represented the calculated
expression level of the unigene in the individual.  This data matrix was used as
an input to EBSeq \citep[v. 1.4.0]{Leng:Bioinformatics:2013}, an R package
designed to identify differentially expressed genes across biological replicates
in a gene expression study. The data were first median-normalized
(\texttt{MedianNorm}).  From these normalized data, we performed 50 iterations
of the \texttt{EBTest} method, which uses an EM algorithm to compute the
posterior probabilities of a unigene being differentially expressed across
tissue types. Unigenes in the dataset that were considered as candidates for
differential expression were required to have an expression level of at least
5.0.  This was controlled by the \texttt{QtrmCut} and \texttt{Qtrim} parameters
of EBTest which removes unigenes from the dataset that do not satisfy the
condition $\text{Qtrim}^{\text{th}} \text{quanitile} <= \text{QtrmCut}$.  As
such, the \texttt{Qtrim} and \texttt{QtrmCut} were set to 1.0 and 5.0,
respectively.  Multiple test correction was done using FDR
\citep{benjamini1995controlling} and posterior convergence was diagnosed
manually by examining the individual Alpha, Beta, and P values returned from
test procedure.  These values converged at 33, 32, and 32 generations,
respectively, providing confidence in the estimates for differential expression.
The count matrix used for DGE, in concert with the exported annotations from
Blast2GO was also used to compute GO enrichment for each ontology (i.e., BP, CC,
MF), using topGO \citep{topgo}, by tissue type, for nodes in the GO ontologies
containing at least 5 genes.  Tests for enrichment in GO terms were computed
using Fisher's exact test and controlled using FDR.

All relevant code, including R scripts and IPython notebooks, for both the
analysis and manuscript can be found at
\url{http://www.github.com/cfriedline/black_spruce}.  Raw data files can be
accessed from the iPlant Data Store at
\url{https://de.iplantc.org/de/?type=data&folder=/iplant/home/cfriedline/pub_data/black_spruce}
and the quality-controlled ESTs can be found in NCBI dbEST under accessions
XXX--XXX.

\section*{Results}

A total of 7232 chromatogram files were obtained from sequencing for
two biological replicates (e.g., P32, P40) for two tissue types:
needle and cambium.  Following processing with Phred and filtering by
length, 5996 raw ESTs were combined into a single file for downstream
analysis.  After processing with Seqclean, 5938 ESTs remained; 2842
were trimmed and 58 were removed from the dataset, either by mapping
to \textit{E. coli} (34), low complexity (1), or length/shortq (23).
Assembly resulted in 1945 unigenes with an average coverage of
$\approx{3}$ ESTs ($\pm \approx{10}, range=[1,274]$). The EST assbemly 
is summarized in Table \ref{tab:est}.

We consider unigenes to be expressed if their estimated level of expression was
at least 2.0 across replicates.  We find 255 unigenes exclusively expressed in
cambium, 123 exclusive to needle, and 57 unigenes expressed in both tissues. DGE
was evaluated for 169 unigenes (transcripts) meeting the defined expression
cutoffs for EBSeq, and 35 of these were reported as significant at FDR = 0.05
(Figure \ref{fig:unigene_venn}).  In cambium, 23 of these unigenes were
differentially expressed as were 12 in the needle tissue, though while 20
unigenes were uniquely expressed in cambium, there were none uniquely expressed
in the needles.  The BP annotations for these unigenes are outlined in Table
\ref{tab:dge_bp}.

There were 933 unigenes ($\sim 48\%$) with GO annotations (mean=$6 \pm
5$ annotations/unigene, range=[1, 44]). Of these unigenes with GO
annotations, 767, 600, 800 were usable by the topGO enrichment tests
for the BP, CC, and MF ontologies, respectively.  The top 20 terms for
each ontology which were signficantly ($q < 0.05$) enriched for each
ontology are displayed in Figure \ref{fig:go_full_combined_20_bh}.  In
addition to the overall summarization of the significantly enriched GO
terms, Figure \ref{fig:go_full_20_bh} shows the top 20 terms (at most) 
which were signficantly enriched, by tissue type.


% \section*{Discussion}

% \section*{Acknowledgements}

\clearpage

\singlespacing
\bibliographystyle{spbasic}
\bibliography{refs}

\clearpage
%figures

\begin{figure}[t]
  \centering
  \includegraphics[width=\textwidth,keepaspectratio]{unigene_venn.pdf}
  \caption{Top: the raw number of expressed unigenes across tissues.  The count
    represents the number of unigenes with expression levels (from RSEM) of at
    least 2.0 in either a single tissue type or across both tissue types.
    Bottom: The patterns of differential expression by tissue.  The count
    represents the number of differentially expressed genes that were found in a
    specific tissue or across both.  Note that the 0 for needle indicates that
    there were no differentially expressed unigenes found exclusively in needle
    tissue and not that there were 0 differentially expressed unigenes in
    needle.}
  \label{fig:go_venn}
\end{figure}



\begin{figure}[t]
  \centering
  \includegraphics[height=8.5in,keepaspectratio]{go_full_combined_20_bh}
  \caption{Top 20 Significantly enriched ($q < 0.05$) GO terms for
    each ontology. The total number of significant terms for each
    ontology is shown in parentheses.}
  \label{fig:go_full_combined_20_bh}
\end{figure}


\begin{figure}[t]
  \centering
  \includegraphics[width=\textwidth]{go_full_20_bh}
  \caption{Top 20 Significantly enriched ($q < 0.05$) GO terms for
    each ontology and tissue type. The total number of significant
    terms for each ontology and tissue type is shown in parentheses}
  \label{fig:go_full_20_bh}
\end{figure}

\clearpage

%tables

\begin{table}[t]
  \caption{EST summary information. The number of reads passing each step 
    of filtering and cleaning as well as the raw reads are reported. Singletons 
    represent the number of unigenes represented by only a single sequence from 
    a sample. The length distribution is reported for the reads passing seqclean 
    filtering.}
  
  \centering
  \begin{tabular}{l}
    \toprule
    Moved to \scriptsize{\url{https://drive.google.com/open?id=1HhuN-pByCpArPCZjZhuKLOwj6JZdmE88jlPboKAodFM&authuser=0}} \\
    \bottomrule
  \end{tabular}
   
\label{tab:est}
  
\end{table}

\begin{table}[t]
\centering
\caption{Biological Process annotations across the 35 differentially expressed 
unigenes for each tissue type. The counts represent the number of unigenes 
having a particular annotation}

\begin{tabular}{lll}
\toprule
    Moved to \scriptsize{\url{https://drive.google.com/open?id=1HhuN-pByCpArPCZjZhuKLOwj6JZdmE88jlPboKAodFM&authuser=0}} \\
\bottomrule
\end{tabular}
\label{tab:dge_bp}
\end{table}

\clearpage

\beginsupplement

\section*{Supplementary material}

% %put supplementary figures and tables below here
\subsection*{Supplementary Figures}\label{ss:supp-fig}

\begin{figure}[t]
  \centering
  \includegraphics[height=8.5in,keepaspectratio]{go_full_combined_20}
  \caption{Top 20 Significantly enriched ($p < 0.05$) GO terms for
    each ontology. The total number of significant terms for each
    ontology is shown in parentheses.}
  \label{fig:go_combined}
\end{figure}


\begin{figure}[t]
  \centering
  \includegraphics[width=\textwidth]{go_full_20}
  \caption{Top 20 Significantly enriched ($p < 0.05$) GO terms for
    each ontology and tissue type. The total number of significant
    terms for each ontology and tissue type is shown in parentheses}
  \label{fig:go_tissue}
\end{figure}
 
\clearpage

\subsection*{Supplementary Tables}\label{ss:supp-tab}
\scriptsize
\begin{longtable}{llllrl}
\caption{Significant GO categories for BP ontology in the cambium tissue. BH indicates which of the topGO classic Fisher p-values $< 0.05$ passed correction at FDR = 0.05.}\\
\label{tab:go-cambium-BP}\\
\toprule
GO.ID & Tissue & Ontology & Term & p-value & BH \\
\midrule
GO:0006030 & cambium & BP &                      chitin metabolic process  & 0.000031 &    True \\
GO:0006032 & cambium & BP &                      chitin catabolic process  & 0.000031 &    True \\
GO:0046348 & cambium & BP &                 amino sugar catabolic process  & 0.000031 &    True \\
GO:1901071 & cambium & BP &   glucosamine-containing compound metaboli...  & 0.000031 &    True \\
GO:1901072 & cambium & BP &   glucosamine-containing compound cataboli...  & 0.000031 &    True \\
GO:0006026 & cambium & BP &                 aminoglycan catabolic process  & 0.000058 &    True \\
GO:0006040 & cambium & BP &                 amino sugar metabolic process  & 0.000058 &    True \\
GO:0016998 & cambium & BP &   cell wall macromolecule catabolic proces...  & 0.000058 &    True \\
GO:0006022 & cambium & BP &                 aminoglycan metabolic process  & 0.000100 &    True \\
GO:0009415 & cambium & BP &                             response to water  & 0.000190 &    True \\
GO:0009057 & cambium & BP &               macromolecule catabolic process  & 0.000290 &    True \\
GO:0010035 & cambium & BP &               response to inorganic substance  & 0.000330 &    True \\
GO:0044036 & cambium & BP &   cell wall macromolecule metabolic proces...  & 0.000660 &    True \\
GO:0006950 & cambium & BP &                            response to stress  & 0.000690 &    True \\
GO:0042221 & cambium & BP &                          response to chemical  & 0.000740 &    True \\
GO:0001101 & cambium & BP &                              response to acid  & 0.000930 &    True \\
GO:1901700 & cambium & BP &        response to oxygen-containing compound  & 0.002420 &   False \\
GO:0009408 & cambium & BP &                              response to heat  & 0.003310 &   False \\
GO:0048569 & cambium & BP &              post-embryonic organ development  & 0.003830 &   False \\
GO:0071554 & cambium & BP &          cell wall organization or biogenesis  & 0.005240 &   False \\
GO:0009628 & cambium & BP &                  response to abiotic stimulus  & 0.012530 &   False \\
GO:0010118 & cambium & BP &                             stomatal movement  & 0.016030 &   False \\
GO:1901136 & cambium & BP &   carbohydrate derivative catabolic proces...  & 0.018900 &   False \\
GO:0048523 & cambium & BP &       negative regulation of cellular process  & 0.021380 &   False \\
GO:0042592 & cambium & BP &                           homeostatic process  & 0.024710 &   False \\
GO:0000278 & cambium & BP &                            mitotic cell cycle  & 0.025790 &   False \\
GO:0043243 & cambium & BP &   positive regulation of protein complex d...  & 0.025790 &   False \\
GO:0051130 & cambium & BP &   positive regulation of cellular componen...  & 0.025790 &   False \\
GO:0051239 & cambium & BP &   regulation of multicellular organismal p...  & 0.025790 &   False \\
GO:0065008 & cambium & BP &              regulation of biological quality  & 0.029860 &   False \\
GO:1901565 & cambium & BP &   organonitrogen compound catabolic proces...  & 0.031290 &   False \\
GO:0009653 & cambium & BP &            anatomical structure morphogenesis  & 0.037480 &   False \\
GO:0044702 & cambium & BP &          single organism reproductive process  & 0.037480 &   False \\
GO:0050896 & cambium & BP &                          response to stimulus  & 0.037790 &   False \\
GO:0007049 & cambium & BP &                                    cell cycle  & 0.038450 &   False \\
GO:0021700 & cambium & BP &                      developmental maturation  & 0.038450 &   False \\
GO:0022411 & cambium & BP &                cellular component disassembly  & 0.038450 &   False \\
GO:0032984 & cambium & BP &            macromolecular complex disassembly  & 0.038450 &   False \\
GO:0043241 & cambium & BP &                   protein complex disassembly  & 0.038450 &   False \\
GO:0043244 & cambium & BP &   regulation of protein complex disassembl...  & 0.038450 &   False \\
GO:0043624 & cambium & BP &          cellular protein complex disassembly  & 0.038450 &   False \\
GO:0001558 & cambium & BP &                     regulation of cell growth  & 0.038450 &   False \\
GO:0007019 & cambium & BP &                  microtubule depolymerization  & 0.038450 &   False \\
GO:0007346 & cambium & BP &              regulation of mitotic cell cycle  & 0.038450 &   False \\
GO:0008283 & cambium & BP &                            cell proliferation  & 0.038450 &   False \\
GO:0009819 & cambium & BP &                              drought recovery  & 0.038450 &   False \\
GO:0009856 & cambium & BP &                                   pollination  & 0.038450 &   False \\
GO:0009860 & cambium & BP &                            pollen tube growth  & 0.038450 &   False \\
GO:0009887 & cambium & BP &                           organ morphogenesis  & 0.038450 &   False \\
GO:0009892 & cambium & BP &      negative regulation of metabolic process  & 0.038450 &   False \\
GO:0009932 & cambium & BP &                               cell tip growth  & 0.038450 &   False \\
GO:0010638 & cambium & BP &   positive regulation of organelle organiz...  & 0.038450 &   False \\
GO:0031109 & cambium & BP &   microtubule polymerization or depolymeri...  & 0.038450 &   False \\
GO:0031110 & cambium & BP &   regulation of microtubule polymerization...  & 0.038450 &   False \\
GO:0031112 & cambium & BP &   positive regulation of microtubule polym...  & 0.038450 &   False \\
GO:0031114 & cambium & BP &   regulation of microtubule depolymerizati...  & 0.038450 &   False \\
GO:0031117 & cambium & BP &   positive regulation of microtubule depol...  & 0.038450 &   False \\
GO:0032886 & cambium & BP &       regulation of microtubule-based process  & 0.038450 &   False \\
GO:0035264 & cambium & BP &                 multicellular organism growth  & 0.038450 &   False \\
GO:0040008 & cambium & BP &                          regulation of growth  & 0.038450 &   False \\
GO:0040014 & cambium & BP &   regulation of multicellular organism gro...  & 0.038450 &   False \\
GO:0043086 & cambium & BP &   negative regulation of catalytic activit...  & 0.038450 &   False \\
GO:0044092 & cambium & BP &   negative regulation of molecular functio...  & 0.038450 &   False \\
GO:0044703 & cambium & BP &           multi-organism reproductive process  & 0.038450 &   False \\
GO:0044706 & cambium & BP &          multi-multicellular organism process  & 0.038450 &   False \\
GO:0048527 & cambium & BP &                      lateral root development  & 0.038450 &   False \\
GO:0048528 & cambium & BP &               post-embryonic root development  & 0.038450 &   False \\
GO:0048610 & cambium & BP &   cellular process involved in reproductio...  & 0.038450 &   False \\
GO:0048768 & cambium & BP &                     root hair cell tip growth  & 0.038450 &   False \\
GO:0048868 & cambium & BP &                       pollen tube development  & 0.038450 &   False \\
GO:0051261 & cambium & BP &                      protein depolymerization  & 0.038450 &   False \\
GO:0051495 & cambium & BP &   positive regulation of cytoskeleton orga...  & 0.038450 &   False \\
GO:0070507 & cambium & BP &   regulation of microtubule cytoskeleton o...  & 0.038450 &   False \\
GO:0090332 & cambium & BP &                              stomatal closure  & 0.038450 &   False \\
GO:0090333 & cambium & BP &                regulation of stomatal closure  & 0.038450 &   False \\
GO:1901879 & cambium & BP &        regulation of protein depolymerization  & 0.038450 &   False \\
GO:1901881 & cambium & BP &   positive regulation of protein depolymer...  & 0.038450 &   False \\
GO:0048522 & cambium & BP &       positive regulation of cellular process  & 0.045520 &   False \\
GO:0044248 & cambium & BP &                    cellular catabolic process  & 0.048520 &   False \\
\bottomrule
\end{longtable}
  
\clearpage
\begin{longtable}{llllrlrr}
\caption{Significant GO categories for MF ontology in the cambium tissue. BH indicates which of the topGO classic Fisher p-values $< 0.05$ passed correction at FDR = 0.05.}\\
\label{tab:go-cambium-MF}\\
\toprule
GO.ID & Tissue & Ontology & Term & p-value & BH & Cambium & Needle \\
\midrule
GO:0030246 & cambium & MF &   carbohydrate binding  & 0.000040 &   True  & 36.73 & 0.0 \\ 
GO:0004568 & cambium & MF &   chitinase activity  & 0.000083 &   True  & 60.0 & 1.0 \\ 
GO:0008061 & cambium & MF &   chitin binding  & 0.000670 &   True  & 55.0 & 0.0 \\ 
GO:0004553 & cambium & MF &   hydrolase activity, hydrolyzing O-glycos...  & 0.000830 &   True  & 6.0 & 6.0 \\ 
GO:0016798 & cambium & MF &   hydrolase activity, acting on glycosyl b...  & 0.000830 &   True  & 0 & 0 \\
GO:0008289 & cambium & MF &   lipid binding  & 0.028270 &   False  & 33.0 & 58.0 \\ 
GO:0005488 & cambium & MF &   binding  & 0.047920 &   False  & 2.0 & 0.0 \\ 
\bottomrule
\end{longtable}

\clearpage
\begin{longtable}{llllrlrr}
\caption{Significant GO categories for CC ontology in the cambium tissue. BH indicates which of the topGO classic Fisher p-values $< 0.05$ passed correction at FDR = 0.05.}\\
\label{tab:go-cambium-CC}\\
\toprule
GO.ID & Tissue & Ontology & Term & p-value & BH & Cambium & Needle \\
\midrule
GO:0005634 & cambium & CC &   nucleus  & 0.000690 &   False  & 77.73 & 35.77 \\ 
GO:0070469 & cambium & CC &   respiratory chain  & 0.018070 &   False  & 1.0 & 0.0 \\ 
GO:0044429 & cambium & CC &   mitochondrial part  & 0.019100 &   False  & 0 & 0 \\
GO:0005737 & cambium & CC &   cytoplasm  & 0.021630 &   False  & 44.73 & 27.0 \\ 
GO:0043226 & cambium & CC &   organelle  & 0.028300 &   False  & 0 & 0 \\
GO:0043229 & cambium & CC &   intracellular organelle  & 0.028300 &   False  & 0 & 0 \\
GO:0042995 & cambium & CC &   cell projection  & 0.032740 &   False  & 0 & 0 \\
GO:0090406 & cambium & CC &   pollen tube  & 0.032740 &   False  & 14.0 & 0.0 \\ 
\bottomrule
\end{longtable}

\clearpage
\begin{longtable}{llllrl}
\caption{Significant GO categories for BP ontology in the needle tissue. BH indicates which of the topGO classic Fisher p-values $< 0.05$ passed correction at FDR = 0.05.}\\
\label{tab:go-needle-BP}\\
\toprule
GO.ID & Tissue & Ontology & Term & p-value & BH \\
\midrule
GO:0015979 & needle & BP &   photosynthesis  & 0.000000 &   True \\
GO:0009853 & needle & BP &   photorespiration  & 0.000000 &   True \\
GO:0043094 & needle & BP &   cellular metabolic compound salvage  & 0.000000 &   True \\
GO:0046487 & needle & BP &   glyoxylate metabolic process  & 0.000000 &   True \\
GO:0015977 & needle & BP &   carbon fixation  & 0.000000 &   True \\
GO:0006081 & needle & BP &   cellular aldehyde metabolic process  & 0.000000 &   True \\
GO:0032787 & needle & BP &   monocarboxylic acid metabolic process  & 0.000001 &   True \\
GO:0055114 & needle & BP &   oxidation-reduction process  & 0.000001 &   True \\
GO:0006082 & needle & BP &   organic acid metabolic process  & 0.000086 &   True \\
GO:0019752 & needle & BP &   carboxylic acid metabolic process  & 0.000086 &   True \\
GO:0043436 & needle & BP &   oxoacid metabolic process  & 0.000086 &   True \\
GO:0018298 & needle & BP &   protein-chromophore linkage  & 0.000150 &   True \\
GO:0019684 & needle & BP &   photosynthesis, light reaction  & 0.000170 &   True \\
GO:0044281 & needle & BP &   small molecule metabolic process  & 0.000230 &   True \\
GO:0006091 & needle & BP &   generation of precursor metabolites and ...  & 0.000650 &   True \\
GO:0015976 & needle & BP &   carbon utilization  & 0.001860 &   False \\
GO:0006952 & needle & BP &   defense response  & 0.002020 &   False \\
GO:0019253 & needle & BP &   reductive pentose-phosphate cycle  & 0.003430 &   False \\
GO:0019685 & needle & BP &   photosynthesis, dark reaction  & 0.003430 &   False \\
GO:0044710 & needle & BP &   single-organism metabolic process  & 0.003880 &   False \\
GO:0009607 & needle & BP &   response to biotic stimulus  & 0.004010 &   False \\
GO:0006464 & needle & BP &   cellular protein modification process  & 0.004030 &   False \\
GO:0036211 & needle & BP &   protein modification process  & 0.004030 &   False \\
GO:0009767 & needle & BP &   photosynthetic electron transport chain  & 0.005500 &   False \\
GO:0009069 & needle & BP &   serine family amino acid metabolic proce...  & 0.007230 &   False \\
GO:0009765 & needle & BP &   photosynthesis, light harvesting  & 0.007230 &   False \\
GO:0016051 & needle & BP &   carbohydrate biosynthetic process  & 0.007380 &   False \\
GO:0006833 & needle & BP &   water transport  & 0.009920 &   False \\
GO:0042044 & needle & BP &   fluid transport  & 0.009920 &   False \\
GO:0044699 & needle & BP &   single-organism process  & 0.012410 &   False \\
GO:0006534 & needle & BP &   cysteine metabolic process  & 0.015700 &   False \\
GO:0009070 & needle & BP &   serine family amino acid biosynthetic pr...  & 0.015700 &   False \\
GO:0019344 & needle & BP &   cysteine biosynthetic process  & 0.015700 &   False \\
GO:0080170 & needle & BP &   hydrogen peroxide transmembrane transpor...  & 0.015700 &   False \\
GO:0043412 & needle & BP &   macromolecule modification  & 0.018610 &   False \\
GO:0010207 & needle & BP &   photosystem II assembly  & 0.019660 &   False \\
GO:0043207 & needle & BP &   response to external biotic stimulus  & 0.033070 &   False \\
GO:0051704 & needle & BP &   multi-organism process  & 0.033070 &   False \\
GO:0051707 & needle & BP &   response to other organism  & 0.033070 &   False \\
GO:0098542 & needle & BP &   defense response to other organism  & 0.034770 &   False \\
GO:0042742 & needle & BP &   defense response to bacterium  & 0.036400 &   False \\
GO:0009735 & needle & BP &   response to cytokinin  & 0.037850 &   False \\
GO:0006730 & needle & BP &   one-carbon metabolic process  & 0.039930 &   False \\
GO:0022900 & needle & BP &   electron transport chain  & 0.039930 &   False \\
GO:0009617 & needle & BP &   response to bacterium  & 0.046980 &   False \\
GO:0006544 & needle & BP &   glycine metabolic process  & 0.049180 &   False \\
GO:0065003 & needle & BP &   macromolecular complex assembly  & 0.049470 &   False \\
\bottomrule
\end{longtable}
  
\clearpage
\begin{longtable}{llllrl}
\caption{Significant GO categories for MF ontology in the needle tissue. BH indicates which of the topGO classic Fisher p-values $< 0.05$ passed correction at FDR = 0.05.}\\
\label{tab:go-needle-MF}\\
\toprule
GO.ID & Tissue & Ontology & Term & p-value & BH & Cambium & Needle \\
\midrule
GO:0016984 & needle & MF &   ribulose-bisphosphate carboxylase activi...  & 0.000000 &   True  & 12.0 & 331.75 \\ 
GO:0016831 & needle & MF &   carboxy-lyase activity  & 0.000000 &   True  & 0 & 0 \\
GO:0016829 & needle & MF &   lyase activity  & 0.000000 &   True  & 3.0 & 9.2 \\ 
GO:0004497 & needle & MF &   monooxygenase activity  & 0.000000 &   True  & 14.0 & 349.95 \\ 
GO:0016830 & needle & MF &   carbon-carbon lyase activity  & 0.000000 &   True  & 0 & 0 \\
GO:0016168 & needle & MF &   chlorophyll binding  & 0.000007 &   True  & 10.0 & 102.0 \\ 
GO:0016491 & needle & MF &   oxidoreductase activity  & 0.000025 &   True  & 6.0 & 49.0 \\ 
GO:0046906 & needle & MF &   tetrapyrrole binding  & 0.000390 &   True  & 0 & 0 \\
GO:0000287 & needle & MF &   magnesium ion binding  & 0.002940 &   False  & 9.0 & 65.06 \\ 
GO:0016835 & needle & MF &   carbon-oxygen lyase activity  & 0.004230 &   False  & 0 & 0 \\
GO:0004089 & needle & MF &   carbonate dehydratase activity  & 0.009920 &   False  & 1.0 & 16.0 \\ 
GO:0005372 & needle & MF &   water transmembrane transporter activity  & 0.009920 &   False  & 0 & 0 \\
GO:0015250 & needle & MF &   water channel activity  & 0.009920 &   False  & 1.0 & 31.77 \\ 
GO:0015267 & needle & MF &   channel activity  & 0.009920 &   False  & 0 & 0 \\
GO:0022803 & needle & MF &   passive transmembrane transporter activi...  & 0.009920 &   False  & 0 & 0 \\
GO:0022838 & needle & MF &   substrate-specific channel activity  & 0.009920 &   False  & 0 & 0 \\
GO:0009055 & needle & MF &   electron carrier activity  & 0.013040 &   False  & 3.0 & 43.0 \\ 
GO:0016836 & needle & MF &   hydro-lyase activity  & 0.017560 &   False  & 0 & 0 \\
GO:0051539 & needle & MF &   4 iron, 4 sulfur cluster binding  & 0.038940 &   False  & 3.0 & 30.0 \\ 
GO:0097159 & needle & MF &   organic cyclic compound binding  & 0.043740 &   False  & 0 & 0 \\
GO:1901363 & needle & MF &   heterocyclic compound binding  & 0.043740 &   False  & 0 & 0 \\
\bottomrule
\end{longtable}

\clearpage
\begin{longtable}{llllrl}
\caption{Significant GO categories for CC ontology in the needle tissue. BH indicates which of the topGO classic Fisher p-values $< 0.05$ passed correction at FDR = 0.05.}\\
\label{tab:go-needle-CC}\\
\toprule
GO.ID & Tissue & Ontology & Term & p-value & BH \\
\midrule
GO:0009573 & needle & CC &   chloroplast ribulose bisphosphate carbox...  & 0.000000 &   True \\
GO:0048492 & needle & CC &   ribulose bisphosphate carboxylase comple...  & 0.000000 &   True \\
GO:0009536 & needle & CC &   plastid  & 0.000000 &   True \\
GO:0009507 & needle & CC &   chloroplast  & 0.000000 &   True \\
GO:0044435 & needle & CC &   plastid part  & 0.000000 &   True \\
GO:0044434 & needle & CC &   chloroplast part  & 0.000000 &   True \\
GO:1902494 & needle & CC &   catalytic complex  & 0.000000 &   True \\
GO:0009532 & needle & CC &   plastid stroma  & 0.000000 &   True \\
GO:0009570 & needle & CC &   chloroplast stroma  & 0.000000 &   True \\
GO:0043234 & needle & CC &   protein complex  & 0.000007 &   True \\
GO:0044444 & needle & CC &   cytoplasmic part  & 0.000036 &   True \\
GO:0009521 & needle & CC &   photosystem  & 0.000042 &   True \\
GO:0043227 & needle & CC &   membrane-bounded organelle  & 0.000140 &   True \\
GO:0043231 & needle & CC &   intracellular membrane-bounded organelle  & 0.000140 &   True \\
GO:0044436 & needle & CC &   thylakoid part  & 0.000170 &   True \\
GO:0009579 & needle & CC &   thylakoid  & 0.000340 &   True \\
GO:0034357 & needle & CC &   photosynthetic membrane  & 0.000510 &   True \\
GO:0009522 & needle & CC &   photosystem I  & 0.000990 &   True \\
GO:0043226 & needle & CC &   organelle  & 0.001200 &   True \\
GO:0043229 & needle & CC &   intracellular organelle  & 0.001200 &   True \\
GO:0044422 & needle & CC &   organelle part  & 0.001330 &   True \\
GO:0044446 & needle & CC &   intracellular organelle part  & 0.001330 &   True \\
GO:0009534 & needle & CC &   chloroplast thylakoid  & 0.002550 &   True \\
GO:0031976 & needle & CC &   plastid thylakoid  & 0.002550 &   True \\
GO:0031984 & needle & CC &   organelle subcompartment  & 0.002550 &   True \\
GO:0042651 & needle & CC &   thylakoid membrane  & 0.008990 &   True \\
GO:0005737 & needle & CC &   cytoplasm  & 0.009180 &   True \\
GO:0009523 & needle & CC &   photosystem II  & 0.010060 &   True \\
GO:0009535 & needle & CC &   chloroplast thylakoid membrane  & 0.011440 &   True \\
GO:0055035 & needle & CC &   plastid thylakoid membrane  & 0.011440 &   True \\
GO:0005622 & needle & CC &   intracellular  & 0.019080 &   False \\
GO:0016021 & needle & CC &   integral component of membrane  & 0.020620 &   False \\
GO:0032991 & needle & CC &   macromolecular complex  & 0.023580 &   False \\
GO:0044425 & needle & CC &   membrane part  & 0.026250 &   False \\
GO:0031224 & needle & CC &   intrinsic component of membrane  & 0.027940 &   False \\
GO:0044424 & needle & CC &   intracellular part  & 0.046640 &   False \\
\bottomrule
\end{longtable}



\end{document}