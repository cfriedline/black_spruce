\documentclass[11pt]{article}
\pdfoutput=1
\usepackage{simplemargins}
\usepackage{url}
\usepackage[pdftex]{graphicx}
\usepackage{setspace}
\graphicspath{{figures/}}
\usepackage{siunitx}
\setlength{\parindent}{0pt} 
\setlength{\parskip}{1.6ex} 
\setallmargins{1in} 
\linespread{1.6}
\usepackage[round]{natbib}
\usepackage{color}
\usepackage{subfigure}
\usepackage{booktabs}
\usepackage{pdflscape}
\usepackage{tabularx}
\usepackage[colorlinks=false,breaklinks]{hyperref}
\listfiles

% make subfigure labels capitalized
\renewcommand{\thesubfigure}{(\Alph{subfigure})}

%setup supplement
\newcommand{\beginsupplement}{%
        \setcounter{table}{0}
        \renewcommand{\thetable}{S\arabic{table}}
        \setcounter{figure}{0}
        \renewcommand{\thefigure}{S\arabic{figure}}
        \renewcommand{\thesection}{S\arabic{section}}
        \renewcommand{\thesubsection}{S\arabic{subsection}} 
     }

\begin{document}

% title must be 150 characters or less
% \begin{flushleft} 
% \singlespacing
% {\large \textbf{title}}

% insert author names, affiliations and corresponding author email.
% author1$^{1}$, 
% author2$^{2}$,


% \bf{1} affil 1
% \\
% \bf{2} affil 2
% \\

% $\ast$ e-mail: corresponding\_author@domain.com
% \end{flushleft}

% \section*{Abstract}

% \section*{Introduction}

\section*{Materials and methods}\label{ss:mats}
A total of 7232 chromatogram files from two sets of replicate
samples, two each for needle and cambium were processed using an
IPython \citep[v 2.1]{per-gra:2007} notebook and various utilities.
The chromatograms were processed using Phred \citep[v.\
020425]{ewing1998b} into Fasta sequences with a trimming cutoff
probability of 0.01 (e.g., Phred score = 20), and sequences which were
less than 100 bases long were excluded.  The ESTs were combined in
multiple ways to facilitate analysis: single files according to tissue
source and replicate name (e.g., P32C, P32N, P40C, P40N), by tissue
source (e.g, cambium and needle), and into single file (where all
sequence names indicated the source and replicate).  Each of these
combined files was processed using SeqClean \citep[v.\ 2/2011][default
parameters]{citeulike:1911083}, screening both for vectors using the
UniVec \citep{cochrane01012010} database and contamination from
\textit{E. coli} K-12 substr.\ DH10B.  The cleaned and trimmed ESTs
were in each case were assembled into unigenes using iAssembler
\citep[v.\ 1.3.2][default parameters]{zheng2011iassembler}.

The assembled unigenes were aligned to a local copy of the nr database
(downloaded in February 2014) with BLASTX \citep{citeulike:238188}
keeping, at most, the top 10 hits with an e-value cutoff of
$e^{-5}$. The resulting alignment XML files were filtered using
BioPython \citep[v.\ 1.6.4]{citeulike:4202607} to include all hits
which were at least 50\% of both unigene length and identity. These
filtered BLAST results were imported into Blast2GO Pro \citep[v.\
2.7.2, bg2\_may14]{citeulike:2733895} for functional annotation.
Annotations were assigned using default parameters for the annotation
step (e-value: $e^{-6}$, annotation cutoff: 55, go weight: 5, HSP-hit
coverage cutoff: 0, taxonomy filtering: none) and evidence code
weights. The full suite of InterPro \citep{citeulike:12942060}
mappings were performed for each unigene with a valid BLAST result and
these results were merged with the GO annotations.  These annotations
were augmented with ANNEX \citep{annex} and mapped to enzyme codes and
KEGG \citep{citeulike:9172127} pathways. The resulting GO annotations 
were then exported to a text file for further analysis.

Differential gene (unigene) expression (DGE) was computed in R
\citep[v. 3.1.1]{R} using RSEM
\citep[v. 1.2.15]{Li:BmcBioinformatics:2011} in the standard way using
bowtie \citep[v. 2.2.3]{Langmead:NatMethods:2012} by first creating a
reference (\texttt{rsem-prepare-reference}) and then the computing the
expression on both a parent-wise and tissue-wise basis
(\texttt{rsem-calculate-expression}). Models were evaluated using
\texttt{rsem-plot-model} and the parent-wise gene results were
combined for further analysis
(\texttt{rsem-generate-data-matrix}). The resulting data matrix was
used in an EBSeq \citep{Leng:Bioinformatics:2013} pipeline to compute
differential gene expression (50 rounds, accounting for differences in
library size using median normalization), considering all unigenes
($Qtrm=1$) to which at least five cleaned ESTs mapped
($QtrmCut=5$). Multiple testing was controlled by FDR
\citep{benjamini1995controlling} at 0.05 using
\texttt{rsem-control-fdr}.  The count matrix used for DGE, in concert
with the exported annotations from Blast2GO was also used to compute
GO enrichment for each ontology (i.e., BP, CC, MF), using topGO
\citep{topgo}, by tissue type, for nodes in the GO ontologies containing 
at least 5 genes.  Tests for enrichment in GO were computed using 
Fisher's exact test.

All relevant code for both the analysis and manuscript can be found at
\url{http://www.github.com/cfriedline/black_spruce}

\section*{Results}

A total of 7232 chromatogram files were obtained from sequencing for
two biological replicates (e.g., P32, P40) for two tissue types:
needle and cambium.  Following processing with Phred and filtering by
length, 5996 raw ESTs were combined into a single file for downstream
analysis.  After processing with Seqclean, 5938 ESTs remained; 2842
were trimmed and 58 were removed from the dataset, either by mapping
to \textit{E. coli} (34), low complexity (1), or length/shortq (23).
Assembly resulted in 1945 unigenes with an average coverage of
$\approx{3}$ ESTs ($\pm \approx{10}, range=[1,274]$). After mapping 
with Blast2GO, 1061 unigenes had some level of associated GO annotations, 
ranging from 1 to 62 ($mean \approx{5} \pm \approx{5}$).

DGE was evaluated for 169 unigenes (transcripts) and 35 of these were
reported as significant at $FDR = 0.05$ (Table \ref{tab:dge_go}). As an additional check,
convergence of the EBTest procedure was confirmed by examining the
parameters Alpha, Beta, and P individually These values converged at
33, 32, and 32 generations, respectively, providing confidence in the
estimates for differential expression.

Of the 1061 unigenes with GO annotations, 848 were usable by the 
topGO enrichment tests, and these results are summarized in Table \ref{tab:dge}.  


% \section*{Discussion}

% \section*{Acknowledgements}

\clearpage

\singlespacing
\bibliographystyle{spbasic}
\bibliography{refs}

\clearpage
%tables

\begin{table}[t]
\centering
\begin{tabularx}{\textwidth}{lXcc}
\toprule 
Unigene & GO terms & Tissue & DGE \\ \midrule
UN0050 & GO:0046872 &  C   & C \\
UN0084 & GO:0004478, GO:0006555, GO:0009651, GO:0009809, GO:0005618, GO:0005730, GO:0005829, GO:0006730, GO:0046872, GO:0006556, GO:0046686, GO:0009409, GO:0009506, GO:0005886, GO:0005524, GO:0071281 & B   & N \\
UN0214 & GO:0006950, GO:0009415 &  C   & C \\
UN0225 & GO:0009507   &  B   & N \\
UN0243 & GO:0016984, GO:0004497, GO:0009853, GO:0055114, GO:0009507, GO:0015979, GO:0015977  & N   & N \\
UN0268 & GO:0016984, GO:0004497, GO:0009853, GO:0055114, GO:0009507, GO:0015979, GO:0015977 & N   & N \\
UN0283 & GO:0055114, GO:0008794 & C   & C \\
UN0285 & GO:0016491, GO:0009055, GO:0000287, GO:0018298, GO:0016168, GO:0051539, GO:0016021, GO:0009522, GO:0015979, GO:0009535, GO:0055114  & N   & N \\
UN0290 & GO:0009507   & N   & N \\
UN0293 & GO:0009507  & B   & N \\
UN0306 & GO:0009512 & B   & N \\
UN0311 & GO:0055114, GO:0008794  & C   & C \\
UN0312 & GO:0055114, GO:0008794 & C   & C \\
UN0326 & GO:0019253, GO:0016984, GO:0004497, GO:0009853, GO:0009507, GO:0055114 & N   & N \\
UN0328 & GO:0020037, GO:0016705, GO:0005506, GO:0055114 & C   & C \\
UN0340 & GO:0005524, GO:0009570 & N   & N \\
UN0350 & GO:0006950, GO:0009415 & C   & C \\
UN0354 & GO:0046872 & C   & C \\
UN0357 & GO:0046872 &  C   & C \\
UN0361 & GO:0016984, GO:0004497, GO:0009853, GO:0055114, GO:0009507, GO:0015979, GO:0015977 & N   & N \\
UN0362 & GO:0004568, GO:0005975, GO:0006032, GO:0008061, GO:0016998 & C   & C \\
UN0638 & GO:0006950, GO:0009415 & C & C \\ 
\bottomrule
\end{tabularx}

\caption{Unigenes exhibiting differential expression across tissues (C = cambium, N = needle, 
B = both needle and cambium) and associated GO terms. The Tissue column represents presence of 
that unigene ($>2$) ESTs in the tissue, and DGE represents the tissue in which the unigene was 
differentially expressed.}
\label{tab:dge_go}
\end{table}


% \begin{table}[t]
% \centering
% \begin{tabular}{lrrr}
% \toprule
% Tissue  & \multicolumn{3}{c}{Ontology} \\ \cmidrule(l){2-4}
%         & BP       & CC      & MP      \\ \midrule
% Cambium & 143      & 95      & 164     \\
% Needle  & 205      & 186     & 204     \\ \bottomrule
% \end{tabular}
% \caption{Number of GO terms enriched for each tissue for each ontology.}
% \label{tab:dge}
% \end{table}


\clearpage
%figures

\begin{figure}[t]
  \centering
  \includegraphics[height=8.5in,keepaspectratio]{go_full_combined_20}
  \caption{Top 20 Significantly enriched ($p < 0.05$) GO terms for
    each ontology. The total number of significant terms for each
    ontology is shown in parentheses.}
  \label{fig:go_tissue}
\end{figure}


\begin{figure}[t]
  \centering
  \includegraphics[width=\textwidth]{go_full_20}
  \caption{Top 20 Significantly enriched ($p < 0.05$) GO terms for
    each ontology and tissue type. The total number of significant
    terms for each ontology is shown in parentheses}
  \label{fig:go_tissue}
\end{figure}


% \clearpage

% \beginsupplement

% \section*{Supplementary material}

% %put supplementary figures and tables below here
% \subsection*{Supplementary Text}\label{ss:supp}


\end{document}

%  LocalWords:  ESTs SeqClean
